\documentclass[a4paper,12pt]{article}
\usepackage{fancyhdr}
\usepackage{fancyheadings}
\usepackage[ngerman,german]{babel}
\usepackage{german}
%\usepackage[latin1]{inputenc}
\usepackage[active]{srcltx}
\usepackage{algorithm}
\usepackage[noend]{algorithmic}
\usepackage{amsmath}
\usepackage{amssymb}
\usepackage{amsthm}
\usepackage{bbm}
\usepackage{enumerate}
\usepackage{graphicx}
\usepackage{ifthen}
\usepackage{listings}
\usepackage{struktex}
\usepackage{hyperref}
\usepackage[breakable]{tcolorbox}
\usepackage[a4paper, left=2cm, right=2cm, top=2cm]{geometry}
\usepackage{mathtools}
\usepackage{tikz}
\usepackage{dsfont}
\usepackage{multicol}
\usepackage{pgfplots}
\usetikzlibrary{trees}
\pgfplotsset{compat=newest}

\tikzstyle{level 1}=[level distance=3.5cm, sibling distance=4.5cm]
\tikzstyle{level 2}=[level distance=3.5cm, sibling distance=2cm]



% Define styles for bags and leafs
\tikzstyle{bag} = [text width=2em, text centered]
\tikzstyle{end} = [circle, minimum width=3pt,fill, inner sep=1pt]

\newcommand{\contradiction}{{\hbox{%
			\setbox0=\hbox{$\mkern-3mu\times\mkern-3mu$}%
			\setbox1=\hbox to0pt{\hss$\times$\hss}%
			\copy0\raisebox{0.5\wd0}{\copy1}\raisebox{-0.5\wd0}{\box1}\box0
}}}

%%%%%%%%%%%%%%%%%%%%%%%%%%%%%%%%%%%%%%%%%%%%%%%%%%%%%%
\newcommand{\Fach}{Grenzwerte}
\newcommand{\Semester}{SoSe 21}
\newcommand{\Uebungsblatt}{Analytische Geometrie} 
\newcommand{\nl}{\\[0,20cm]}
\newcommand{\lnl}{\\[0,30cm]}
\newcommand{\xlnl}{\\[0,75cm]}
%%%%%%%%%%%%%%%%%%%%%%%%%%%%%%%%%%%%%%%%%%%%%%%%%%%%%%


\setlength{\parindent}{0em}
\topmargin -2.0cm
\oddsidemargin 0cm
\evensidemargin 0cm
\setlength{\textheight}{9.6in}
\setlength{\textwidth}{6.9in}
\addtolength{\hoffset}{-22pt}


\newcommand{\limes}[2]{
	\lim\limits_{x\rightarrow #1}\quad  #2
}
\newcommand{\limesh}[1]{
	\lim\limits_{h\rightarrow 0}\quad  #1
}
\newcommand{\limesr}[2]{
	\lim\limits_{\underset{x > #1}{x\rightarrow #1}}\quad  #2
}
\newcommand{\limesl}[2]{
	\lim\limits_{\underset{x < #1}{x\rightarrow #1}}\quad  #2
}
\newcommand{\Aufgabe}[2]{
	{
		\vspace*{0.3cm}
		\begin{tcolorbox}[breakable,colback=yellow!0,colframe=black!65!black,title=\textbf{Aufgabe #1:},width=\linewidth ]
			{#2}
		\end{tcolorbox}
		
		
	}
}
\newcommand{\Hinweis}[1]{
	\vspace*{0.3cm}
	\begin{tcolorbox}[breakable,colback=yellow!10,colframe=yellow!65!black,title=\textbf{Hinweis:},width=\linewidth ]
		{#1}
	\end{tcolorbox}
}
\newcommand{\SHA}[1]{
	\vspace*{0.1cm}
	\begin{tcolorbox}[breakable,colback=blue!5,colframe=blue!65!black,title=\textbf{Richtiger SHA256 Hash:},width=\linewidth ]
		{\texttt{{#1}}}
	\end{tcolorbox}
}
\newcommand{\Beispiel}[1]{
	\vspace*{0.2cm}
	\begin{tcolorbox}[breakable,colback=yellow!0,colframe=green!65!black,title=\textbf{Beispiel:},width=\linewidth ]
		{#1}
	\end{tcolorbox}
}
\newcommand{\p}[2]{\pi_{#2}^{(#1)}}
\newcommand{\eing}[1]{\begin{enumerate}[\quad]
		\item #1
\end{enumerate}}

\newcommand{\abc}[1]{
	\begin{enumerate}[(a)]
		#1
	\end{enumerate}
}
\makeatletter
\newcommand{\Spvek}[2][r]{%
	\gdef\@VORNE{1}
	\left(\hskip-\arraycolsep%
	\begin{array}{#1}\vekSp@lten{#2}\end{array}%
	\hskip-\arraycolsep\right)}

\def\vekSp@lten#1{\xvekSp@lten#1;vekL@stLine;}
\def\vekL@stLine{vekL@stLine}
\def\xvekSp@lten#1;{\def\temp{#1}%
	\ifx\temp\vekL@stLine
	\else
	\ifnum\@VORNE=1\gdef\@VORNE{0}
	\else\@arraycr\fi%
	#1%
	\expandafter\xvekSp@lten
	\fi}
\makeatother
\newcommand{\integral}[4]{\int\limits_{#1}^{#2} {#3} {\quad d #4}}
\newcommand{\summe}[3]{\sum\limits_{#1}^{#2} #3}
\begin{document}
	\thispagestyle{fancy}
	\begin{center}
		\LARGE \sf \textbf{ \Uebungsblatt{}}
	\end{center}
	
	\vspace*{0.1cm}
	\section{Einführung}
	Viele Anwendungsfälle der Mathematik in der heutigen Welt sind geometrische Rechnungen im Mehrdimensionalen Raum. So werde diese in verschiedener Planungssoftware zur Konstruktion von so gut wie allen Dingen heutzutage eingesetzt. Für die Brückenplanung nutzt der Bauingeneur Auto-CAD, zum Planen der neuen Küche manche den IKEA-Küchenplaner und Vectary zum Planen von Drucken für den 3D-Drucker.\\
	Die bisher gelernten Beschreibungsarten für Geometrie sind leider bisher noch nicht so mächtig wie wir sie gerne hätten. So versuche zum Beispiel Eine Gerade im 2 dimensionalen Raum, welche parallel zur y-Achse ist, zu beschreiben. Dies ist mit dem herkömmlichen $y=mx+n$ unmöglich. Gleichzeitig wollen wir nicht nur im 2-Dimensionalen, sondern auch $n$-Dimensionalen Raum arbeiten und dort zum Beispiel den Abstand von 2 diskreten Punken oder Objekten berechnen.\\ 
	\section{Lineare Gleichungssysteme}
	\subsection{Einsetzungsverfahren}
	\subsection{Gleichsetzungsverfahren}
	\subsection{Additionsverfahren}
	\subsection{Gauss-Jordan}
	\section{Vektorarithmetik}
	\subsection{Addition - Subtraktion}
	\subsection{Multiplikation}
	\subsection{Norm}
	\subsection{Skalarmultiplikation}
	\subsection{Lineare Abhängigkeit}
	\section{Geraden}
	\[g: \vec{x}=\vec{s} + t\cdot \vec{r}\]
	\[g: \Spvek{x_x;x_y;x_z}=\Spvek{s_x;s_y;s_z} + t\cdot \Spvek{r_x;r_y;r_z}\]
	\subsection{Konstruktion von Geraden}
	Wie auch schon für $y=mx+n$ können wir solche Geraden anhand von gegebenen Eigenschaften Konstruieren. Solche Eigenschaften können zum Beispiel zwei gegebenen Punkte, welche auf der Gerade liegen sollen, sein. Selterner wird ein Punkt von der Gerade gegeben und ein Richtungsvektor in welche die Gerade verläuft.
	\subsubsection{Gegeben Punkt und Richtung}
	Dies macht uns die konstruktion besonders einfach.
	Sei $P$ der gegebene Punkt, $\vec{r}$ der Richtungsvektor. Die Gerade $g$, welche durch $P$ verläuft mit Richtung $\vec{r}$ wird so konstruiert:
	\[g: \,\vec{x} = \overline{0P} + t\cdot \vec{r}\]
	\subsubsection{Gegeben zwei Punkte}
	Etwas schwieriger wird die Konstruktion mit lediglich 2 gegebenen Punkten.\\
	Sei $A,B$ gegebene Punkte, dann konstruieren wir $g$, indem wir einen der gegebenen Punkte als Stützvektor wählen und den Richtungsvektor gleich dem Vektor, durch den wir von Punkt $A$ zu Punkt $B$ kommen, wählen.
	\[g:\, \vec{x}=\overline{0A} +t\cdot \overline{AB} \]
	\subsection{Lagebeziehung Punkt und Gerade}
	Für einen gegebenen Punkt $P$ und Gerade $g$ kann man entscheiden ob $P$ auf $g$ liegt. Sollte dies nicht der Fall sein ist es weiterhin möglich den Punkt $Q$ auf $g$ anzugeben welcher am nächsten an $P$ liegt.
	\subsubsection{Liegt der Punkt $P$ auf der Gerade?}
	Gegeben sei Punkt $P=(P_x,P_y,P_z)$ und Gerade $g:\, \vec{x}=\Spvek{s_x;s_y;s_z} + t \cdot \Spvek{r_x;r_y;r_z}$.\\
	Um zu überprüfen ob $P$ auf $g$ liegt schaffen wir ein Gleichungssystem:
	\begin{align*}
		\Spvek{P_x;P_y;P_z}&=\Spvek{s_x;s_y;s_z} + t \cdot \Spvek{r_x;r_y;r_z}&&\bigg\vert -\Spvek{s_x;s_y;s_z} \\
		\Spvek{P_x - s_x;P_y-s_y;P_z-s_z} &= t \cdot \Spvek{r_x;r_y;r_z} \qquad \begin{bmatrix}
			t_x=\frac{P_x-s_x}{r_x}\\t_y=\frac{P_y-s_y}{r_y}\\t_z=\frac{P_z-s_z}{r_z}\\
		\end{bmatrix}&&
	\end{align*}
	Falls nun $t_x=t_y=t_z$ gilt, mit anderen Worten die Lösungsmenge des Gleichungssystems nicht leer ist, dann liegt der Punkt auf der Geraden.\\
	Gilt dies nicht, liegt der Punkt P auch nicht auf der Geraden P.
	\subsubsection{Nächste Punkt auf der Geraden}
	\subsection{Lagebeziehungen zwischen Geraden und Geraden}
	Auch kann die Lage zwischen zwei Geraden ermittelt werden, das heißt ob diese sich schneiden, parallel sind, windschief stehen oder aufeinander liegen.\\
	Gegeben Geraden $g:\, \vec{x}_g = \vec{s}_g + t\cdot \vec{r}_g,\quad h:\, \vec{x}_h = \vec{s}_h + k\cdot \vec{r}_h$\\
	Herangehensweise:
	\subsubsection{Auf Schnittpunkt Prüfen}
	Wie bei 2 Geraden der Form $y=mx+n$ müssen wir diese gleichsetzen, also:
	\begin{align*}
		\vec{s}_g + t\cdot \vec{r}_g & = \vec{s}_h + k\cdot \vec{r}_h&&\big \vert -\vec{s}_g,\, -  k\cdot \vec{r}_h\\
		t\cdot \vec{r}_g - k\cdot \vec{r}_h &= \vec{s}_h - \vec{s}_g\\
		t\cdot \Spvek{r_{g_x};r_{g_y};r_{g_z}} - k\cdot \Spvek{r_{h_x};r_{h_y};r_{h_z}} &= \Spvek{s_{h_x}-s_{g_x};s_{h_y}-s_{g_y};s_{h_z}-s_{g_z}}\\
		&\downarrow\\
		r_{g_x}\cdot \color{red}t\color{black} - r_{h_x}\cdot \color{blue}k\color{black} &= s_{h_x}-s_{g_x}\\
		r_{g_y}\cdot \color{red}t\color{black} - r_{h_y}\cdot \color{blue}k\color{black} &= s_{h_y}-s_{g_y}\\
		r_{g_z}\cdot \color{red}t\color{black} - r_{h_z}\cdot \color{blue}k\color{black} &= s_{h_z}-s_{g_z}\\
	\end{align*}
	Dabei sind $\color{blue}k\color{black},\color{red}t\color{black}$ die Variablen. Falls die Lösungsmenge des Gleichungssystems:\begin{enumerate}[(I)]
		\item $|L|=0$ die Geraden schneiden sich nicht
		\item $|L|=1$ die Geraden schneiden sich
		\item $|L|=\infty$ die Geraden liegen aufeinander
	\end{enumerate}
	Für (I) ist nun noch zu prüfen ob die Geraden windschief oder parallel zueinander sind.
	\subsubsection{Auf Lineare Abhängigkeit prüfen}
	 Gelte (I), nun ist lediglich die lineare Abhängigkeit zu prüfen, ein Synonym für Parallelität. Setzte die Richtungsvektoren der Geraden mit Hilfe einer Variable  gleich:
	 \[\Spvek{r_{h_x};r_{h_y};r_{h_z}} = t \cdot  \Spvek{r_{g_x};r_{g_y};r_{g_z}} \qquad \begin{bmatrix}
	 	t_x={r_{h_x}}/{r_{g_x}}\\t_y={r_{h_y}}/{r_{g_y}}\\t_z={r_{h_z}}/{r_{g_z}}
	 \end{bmatrix}\]
 	Falls $t_x=t_y=t_z$ sind die Richtungsvektoren linear abhängig und damit die beiden Geraden parallel.\\ Sonst sind die Geraden windschief.
 	\subsection{Aufgaben}
 	\Aufgabe{4.4.1: (Konstruktion gegeben 2 Punkte)}{
 		
 	}
 	\Aufgabe{4.4.2: (Lagebeziehung Punkt - Gerade)}{
 		
 	}
 	\Aufgabe{4.4.3: (Lagebeziehung Gerade - Gerade)}{
 		
 	}
	\section{Ebenen}
	
\end{document} 
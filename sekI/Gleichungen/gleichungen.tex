\documentclass[a4paper,12pt]{article}
\usepackage{fancyhdr}
\usepackage{fancyheadings}
\usepackage[ngerman,german]{babel}
\usepackage{german}
\usepackage[utf8]{inputenc}
%\usepackage[latin1]{inputenc}
\usepackage[active]{srcltx}
\usepackage{algorithm}
\usepackage[noend]{algorithmic}
\usepackage{amsmath}
\usepackage{amssymb}
\usepackage{amsthm}
\usepackage{bbm}
\usepackage{enumerate}
\usepackage{graphicx}
\usepackage{ifthen}
\usepackage{listings}
\usepackage{struktex}
\usepackage{hyperref}
\usepackage[breakable]{tcolorbox}
\usepackage[a4paper, left=2cm, right=2cm, top=2cm]{geometry}
\usepackage{mathtools}
\usepackage{tikz}
\usepackage{dsfont}
\usetikzlibrary{trees}


\tikzstyle{level 1}=[level distance=3.5cm, sibling distance=4.5cm]
\tikzstyle{level 2}=[level distance=3.5cm, sibling distance=2cm]



% Define styles for bags and leafs
\tikzstyle{bag} = [text width=2em, text centered]
\tikzstyle{end} = [circle, minimum width=3pt,fill, inner sep=1pt]

\newcommand{\contradiction}{{\hbox{%
			\setbox0=\hbox{$\mkern-3mu\times\mkern-3mu$}%
			\setbox1=\hbox to0pt{\hss$\times$\hss}%
			\copy0\raisebox{0.5\wd0}{\copy1}\raisebox{-0.5\wd0}{\box1}\box0
}}}

%%%%%%%%%%%%%%%%%%%%%%%%%%%%%%%%%%%%%%%%%%%%%%%%%%%%%%
\newcommand{\Fach}{Grenzwerte}
\newcommand{\Semester}{SoSe 21}
\newcommand{\Uebungsblatt}{ Lineare Gleichungen und Ungleichungen} 
\newcommand{\nl}{\\[0,20cm]}
\newcommand{\lnl}{\\[0,30cm]}
\newcommand{\xlnl}{\\[0,75cm]}
%%%%%%%%%%%%%%%%%%%%%%%%%%%%%%%%%%%%%%%%%%%%%%%%%%%%%%


\setlength{\parindent}{0em}
\topmargin -2.0cm
\oddsidemargin 0cm
\evensidemargin 0cm
\setlength{\textheight}{9.6in}
\setlength{\textwidth}{6.9in}
\addtolength{\hoffset}{-22pt}


\newcommand{\limes}[2]{
	\lim\limits_{x\rightarrow #1}\quad  #2
}
\newcommand{\limesh}[1]{
	\lim\limits_{h\rightarrow 0}\quad  #1
}
\newcommand{\limesr}[2]{
	\lim\limits_{\underset{x > #1}{x\rightarrow #1}}\quad  #2
}
\newcommand{\limesl}[2]{
	\lim\limits_{\underset{x < #1}{x\rightarrow #1}}\quad  #2
}
\newcommand{\Aufgabe}[2]{
	{
		\vspace*{0.3cm}
		\begin{tcolorbox}[breakable,colback=yellow!0,colframe=black!65!black,title=\textbf{Aufgabe #1:},width=\linewidth ]
			{#2}
		\end{tcolorbox}
		
		
	}
}
\newcommand{\Beispiel}[1]{
	\vspace*{0.3cm}
	\begin{tcolorbox}[breakable,colback=yellow!0,colframe=green!65!black,title=\textbf{Beispiel:},width=\linewidth ]
		{#1}
	\end{tcolorbox}
}
\newcommand{\Heuristik}[1]{
	\vspace*{0.3cm}
	\begin{tcolorbox}[breakable,colback=yellow!0,colframe=red!65!black,title=\textbf{Wie gehe ich vor?:},width=\linewidth ]
		{#1}
	\end{tcolorbox}
}
\newcommand{\p}[2]{\pi_{#2}^{(#1)}}
\newcommand{\eing}[1]{\begin{enumerate}[\quad]
		\item #1
\end{enumerate}}

\newcommand{\abc}[1]{
	\begin{enumerate}[(a)]
		#1
	\end{enumerate}
}

\newcommand{\integral}[4]{\int\limits_{#1}^{#2} {#3} {\quad d #4}}
\newcommand{\summe}[3]{\sum\limits_{#1}^{#2} #3}
\begin{document}
	\thispagestyle{fancy}
	\begin{center}
		\LARGE \sf \textbf{ \Uebungsblatt{}}
	\end{center}

	
	%<<<
	\Aufgabe{1 (Gleichungen)}{
		\Heuristik{
			Gleichungen sind Aussagen über die Gleichheit von Zahlen. In Worten: Zahl 1 ist genau so groß wie Zahl 2.
			Solche Gleichungen können wir verändern, in dem wir Äquivalenzumformungen durchführen. 
			Das heißt in gut Deutsch einfach: Unsere Rechenregeln die wir kennen dürfen wir auf die Gleichung anwenden, müssen aber diese Rechnung immer auf beiden Seiten anwenden. Zum Lösen reicht in der Regel: 
			\begin{enumerate}[$\circ$]
				\item Multiplikation mit einer Zahl oder Variable
				\item Division mit einer Zahl oder Variable
				\item Addition mit einer Zahl oder Variable
				\item Substraktion mit einer Zahl oder Variable
			\end{enumerate}
			\Beispiel{
				\begin{align*}
					4&=4 \qquad\qquad\qquad &| +2\\
					6&=6 \qquad\qquad\qquad &| *3\\
					18&=18 \qquad\qquad\qquad &| :2\\
					9&=9 \qquad\qquad\qquad &| \cdot x\\
					9x&=9x \qquad\qquad\qquad \\
				\end{align*}}
			Man sieht schnell, dass die Gleichheit der Zahlen nie verletzt wird. Also Die Aussage bleibt immer wahr.\\
			Dass können wir jetzt ausnutzen und bei Gleichungen mit einem $x$ diese so lange umstellen bis wir wissen welchen Wert $x$ haben muss.
			Dazu bringen wir die Gleichung in die Form:\[ \textbf{[Zahl 1]}\cdot x = \textbf{[Zahl 2]}\]
			Indem wir in der Ursprungsgleichung nur + und - rechnen.
			Danach teile durch \textbf{[Zahl 1]}
			\begin{align*}
				\underbrace{\frac{\textbf{[Zahl 1]}}{\textbf{[Zahl 1]}}}_{=1}\cdot x = \frac{\textbf{[Zahl 2]}}{\textbf{[Zahl 1]}}\\
				x= \frac{\textbf{[Zahl 2]}}{\textbf{[Zahl 1]}}
			\end{align*}
		}
		\Beispiel{
				\begin{align*}
					3x+3 &= x-1\qquad\qquad\qquad &| -x\\
					2x+3 &= -1\qquad\qquad\qquad &| -3\\
					2x &= -4\qquad\qquad\qquad &| : 2\\
					x&=\frac{-4}{2}= -2
				\end{align*}
	}
		\begin{enumerate}[(a)]
			\item \[4x+4=3x+3\]
			\item \[5x-2=x+6\]
			\item \[3x=x+5\]
			\item \[7x-9=2x+5\]
			\item \[\frac{1}{12}x-5=3\]
			\item \[-8x + 5 = -5\]
			\item \[x+4=9x-(5-x)\]
			\item \[3(4x-3)=4(3x-4)\]
			\item \[3(4x+4)=4(3-4x)\]
			\item \[3(a-4)=1-\frac{1}{5}(2-a)\]
		\end{enumerate}
}

\end{document} 


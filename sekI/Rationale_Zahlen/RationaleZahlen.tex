\documentclass[a4paper,12pt]{article}
\usepackage{fancyhdr}
\usepackage{fancyheadings}
\usepackage[ngerman,german]{babel}
\usepackage{german}
\usepackage[utf8]{inputenc}
%\usepackage[latin1]{inputenc}
\usepackage[active]{srcltx}
\usepackage{algorithm}
\usepackage[noend]{algorithmic}
\usepackage{amsmath}
\usepackage{amssymb}
\usepackage{amsthm}
\usepackage{bbm}
\usepackage{enumerate}
\usepackage{graphicx}
\usepackage{ifthen}
\usepackage{listings}
\usepackage{struktex}
\usepackage{hyperref}
\usepackage[breakable]{tcolorbox}
\usepackage[a4paper, left=2cm, right=2cm, top=2cm]{geometry}
\usepackage{mathtools}
\usepackage{tikz}
\usepackage{dsfont}
\usetikzlibrary{trees}


\tikzstyle{level 1}=[level distance=3.5cm, sibling distance=4.5cm]
\tikzstyle{level 2}=[level distance=3.5cm, sibling distance=2cm]



% Define styles for bags and leafs
\tikzstyle{bag} = [text width=2em, text centered]
\tikzstyle{end} = [circle, minimum width=3pt,fill, inner sep=1pt]

\newcommand{\contradiction}{{\hbox{%
			\setbox0=\hbox{$\mkern-3mu\times\mkern-3mu$}%
			\setbox1=\hbox to0pt{\hss$\times$\hss}%
			\copy0\raisebox{0.5\wd0}{\copy1}\raisebox{-0.5\wd0}{\box1}\box0
}}}

%%%%%%%%%%%%%%%%%%%%%%%%%%%%%%%%%%%%%%%%%%%%%%%%%%%%%%
\newcommand{\Fach}{Rationale Zahlen \quad $\mathds{Q}$}
\newcommand{\Semester}{SoSe 21}
\newcommand{\Uebungsblatt}{ Rationale Zahlen \quad $\mathds{Q}$} 
\newcommand{\nl}{\\[0,20cm]}
\newcommand{\lnl}{\\[0,30cm]}
\newcommand{\xlnl}{\\[0,75cm]}
%%%%%%%%%%%%%%%%%%%%%%%%%%%%%%%%%%%%%%%%%%%%%%%%%%%%%%


\setlength{\parindent}{0em}
\topmargin -2.0cm
\oddsidemargin 0cm
\evensidemargin 0cm
\setlength{\textheight}{9.6in}
\setlength{\textwidth}{6.9in}
\addtolength{\hoffset}{-22pt}



\newcommand{\Aufgabe}[2]{
	{
		\vspace*{0.3cm}
		\begin{tcolorbox}[breakable,colback=yellow!0,colframe=black!65!black,title=\textbf{Aufgabe #1:},width=\linewidth ]
			{#2}
		\end{tcolorbox}
		
		
	}
}
\newcommand{\p}[2]{\pi_{#2}^{(#1)}}
\newcommand{\eing}[1]{\begin{enumerate}[\quad]
		\item #1
\end{enumerate}}

\newcommand{\abc}[1]{
	\begin{enumerate}[(a)]
		#1
	\end{enumerate}
}

\newcommand{\integral}[4]{\int\limits_{#1}^{#2} {#3} {\quad d #4}}
\newcommand{\summe}[3]{\sum\limits_{#1}^{#2} #3}
\begin{document}
	\thispagestyle{fancy}
	\begin{center}
		\LARGE \sf \textbf{\" Ubungsblatt \Uebungsblatt{}}
	\end{center}
	\vspace*{0.1cm}
	
	%<<<
	\Aufgabe{1 (Zuordnung)}{
		Kreise alle Rationalen Zahlen ($\mathds{Q}$) ein.\\\\
		\begin{tabular}{cccccccccc}
			$\frac{1}{2}$&$\frac{-5}{0}$&$\sqrt{2}$&$\pi (= 3,141...)$&$\frac{2}{3}$&$\infty$&$-\frac{31764536897}{232732372}$&$75,125$&$\sqrt{16}$&$\sqrt{2}^2$\\
		\end{tabular}
	}
	\Aufgabe{2 (Zahlenstrahl)}{
		Ordne die Zahlen im Zahlenstrahl ein.
		\begin{center}
			\begin{tikzpicture}
				\draw[<->] (0,0) -- (16cm,0) ;
				\draw (1,0) node[above=1.8]{$-7$};
				\draw (1.5,0) node{\tiny$|$};
				\draw (2.5,0) node{\tiny$|$};
				\draw (3.5,0) node{\tiny$|$};
				\draw (4.5,0) node{\tiny$|$};
				\draw (5.5,0) node{\tiny$|$};
				\draw (6.5,0) node{\tiny$|$};
				\draw (7.5,0) node{\tiny$|$};
				\draw (8.5,0) node{\tiny$|$};
				\draw (9.5,0) node{\tiny$|$};
				\draw (10.5,0) node{\tiny$|$};
				\draw (11.5,0) node{\tiny$|$};
				\draw (12.5,0) node{\tiny$|$};
				\draw (13.5,0) node{\tiny$|$};
				\draw (14.5,0) node{\tiny$|$};
				\draw (1,0) node{\scriptsize$|$};
				\draw (2,0) node{\scriptsize$|$};
				\draw (3,0) node{\scriptsize$|$};
				\draw (4,0) node{\scriptsize$|$};
				\draw (5,0) node{\scriptsize$|$};
				\draw (6,0) node{\scriptsize$|$};
				\draw (7,0) node{\scriptsize$|$};
				\draw (8,0) node{\scriptsize$|$};
				\draw (9,0) node{\scriptsize$|$};
				\draw (10,0) node{\scriptsize$|$};
				\draw (11,0) node{\scriptsize$|$};
				\draw (12,0) node{\scriptsize$|$};
				\draw (13,0) node{\scriptsize$|$};
				\draw (14,0) node{\scriptsize$|$};
				\draw (15,0) node{\scriptsize$|$};
				\draw (2,0) node[above=1.8]{$-6$};
				\draw (3,0) node[above=1.8]{$-5$};
				\draw (4,0) node[above=1.8]{$-4$};
				\draw (5,0) node[above=1.8]{$-3$};
				\draw (6,0) node[above=1.8]{$-2$};
				\draw (7,0) node[above=1.8]{$-1$};
				\draw (8,0) node[above=1.8]{$0$};
				\draw (9,0) node[above=1.8]{$1$};
				\draw (10,0) node[above=1.8]{$2$};
				\draw (11,0) node[above=1.8]{$3$};
				\draw (12,0) node[above=1.8]{$4$};
				\draw (13,0) node[above=1.8]{$5$};
				\draw (14,0) node[above=1.8]{$6$};
				\draw (15,0) node[above=1.8]{$7$};
			\end{tikzpicture}
		\end{center}
\[\]	
		\begin{enumerate}[(a)]
			\item $4.5$	\qquad\quad d) \quad $\frac{16}{4}$ 
			\item $\frac{7}{2}$\quad\quad	\qquad e) \quad $-(- 5)$
			\item $-\frac{8}{4}$	\quad\qquad f) \quad $\frac{51}{100}$
			\item $-6.1$	\qquad g) \quad $\frac{123456}{123456}$
		\end{enumerate}
	}
	\Aufgabe{3 Addition und Subtraktion}{
		\begin{enumerate}[(a)]
			\item $\frac{4}{2} + \frac{1}{3}=$
			\item $-\frac{3}{10} + \frac{15}{50}=$
			\item $\frac{8}{-6} - \frac{2}{3}=$
			\item $\frac{12}{3} - (-\frac{8}{6})=$
			\item $\frac{-21}{30} - \frac{30}{100}=$
			\item $\frac{45}{100} - 3,5=$
			\item $ 8,3 - \frac{1}{2}=$
			\item $-3,4 - 2,223 =$
			\item $5,68 + 4,32=$
			\item $1,11 - (-2,99)=$
			\item $4,25 + 0,55=$
		\end{enumerate}
	}
\end{document} 


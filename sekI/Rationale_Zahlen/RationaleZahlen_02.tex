\documentclass[a4paper,12pt]{article}
\usepackage{fancyhdr}
\usepackage{fancyheadings}
\usepackage[ngerman,german]{babel}
\usepackage{german}
\usepackage[utf8]{inputenc}
%\usepackage[latin1]{inputenc}
\usepackage[active]{srcltx}
\usepackage{algorithm}
\usepackage[noend]{algorithmic}
\usepackage{amsmath}
\usepackage{amssymb}
\usepackage{amsthm}
\usepackage{bbm}
\usepackage{enumerate}
\usepackage{graphicx}
\usepackage{ifthen}
\usepackage{listings}
\usepackage{struktex}
\usepackage{hyperref}
\usepackage[breakable]{tcolorbox}
\usepackage[a4paper, left=2cm, right=2cm, top=2cm]{geometry}
\usepackage{mathtools}
\usepackage{tikz}
\usepackage{dsfont}
\usetikzlibrary{trees}


\tikzstyle{level 1}=[level distance=3.5cm, sibling distance=4.5cm]
\tikzstyle{level 2}=[level distance=3.5cm, sibling distance=2cm]



% Define styles for bags and leafs
\tikzstyle{bag} = [text width=2em, text centered]
\tikzstyle{end} = [circle, minimum width=3pt,fill, inner sep=1pt]

\newcommand{\contradiction}{{\hbox{%
			\setbox0=\hbox{$\mkern-3mu\times\mkern-3mu$}%
			\setbox1=\hbox to0pt{\hss$\times$\hss}%
			\copy0\raisebox{0.5\wd0}{\copy1}\raisebox{-0.5\wd0}{\box1}\box0
}}}

%%%%%%%%%%%%%%%%%%%%%%%%%%%%%%%%%%%%%%%%%%%%%%%%%%%%%%
\newcommand{\Fach}{Rationale Zahlen \quad $\mathds{Q}$}
\newcommand{\Semester}{SoSe 21}
\newcommand{\Uebungsblatt}{ Rationale Zahlen \quad $\mathds{Q}$} 
\newcommand{\nl}{\\[0,20cm]}
\newcommand{\lnl}{\\[0,30cm]}
\newcommand{\xlnl}{\\[0,75cm]}
%%%%%%%%%%%%%%%%%%%%%%%%%%%%%%%%%%%%%%%%%%%%%%%%%%%%%%


\setlength{\parindent}{0em}
\topmargin -2.0cm
\oddsidemargin 0cm
\evensidemargin 0cm
\setlength{\textheight}{9.6in}
\setlength{\textwidth}{6.9in}
\addtolength{\hoffset}{-22pt}



\newcommand{\Aufgabe}[2]{
	{
		\vspace*{0.3cm}
		\begin{tcolorbox}[breakable,colback=yellow!0,colframe=black!65!black,title=\textbf{Aufgabe #1:},width=\linewidth ]
			{#2}
		\end{tcolorbox}
		
		
	}
}
\newcommand{\p}[2]{\pi_{#2}^{(#1)}}
\newcommand{\eing}[1]{\begin{enumerate}[\quad]
		\item #1
\end{enumerate}}

\newcommand{\abc}[1]{
	\begin{enumerate}[(a)]
		#1
	\end{enumerate}
}

\newcommand{\integral}[4]{\int\limits_{#1}^{#2} {#3} {\quad d #4}}
\newcommand{\summe}[3]{\sum\limits_{#1}^{#2} #3}
\begin{document}
	\thispagestyle{fancy}
	\begin{center}
		\LARGE \sf \textbf{\" Ubungsblatt 2 \Uebungsblatt{}}
	\end{center}
	\vspace*{0.1cm}
	
	%<<<
	\Aufgabe{1 (Multiplikation/Division)}{
		\begin{center}
			\begin{align*}
				&\textbf{(a)}\quad  \frac{12}{5}\cdot \frac{10}{3}\qquad\qquad\qquad\qquad\qquad\qquad\qquad\qquad\qquad\qquad\qquad\qquad\qquad\qquad\qquad\\\\
				&\textbf{(b)}\quad  \frac{11}{8}\cdot \frac{56}{33}\qquad \\\\
				&\textbf{(c)}\quad \frac{9}{10}\cdot \frac{8}{7}  \qquad \\\\
				&\textbf{(d)}\quad \frac{4}{7}\cdot \frac{14}{2}\\\\
				&\textbf{(e)}\quad \frac{15}{6}\cdot \frac{2}{9} \qquad \\\\
				&\textbf{(f)}\quad 0\cdot \frac{8}{3} \qquad \\\\
				&\textbf{(g)}\quad \frac{12}{5}: \frac{15}{7} \qquad \\\\
				&\textbf{(h)}\quad\frac{11}{8}: \frac{55}{3}\\\\
				&\textbf{(i)}\quad \frac{9}{10}: \frac{3}{7}\qquad \\\\
				&\textbf{(j)}\quad \frac{7}{3}: \frac{14}{9}\\\\
			\end{align*}
		\end{center}
	}

\end{document} 


\documentclass[a4paper,12pt]{article}
\usepackage{fancyhdr}
\usepackage{fancyheadings}
\usepackage[german]{babel}
\usepackage{german}

\usepackage[utf8]{inputenc}
%\usepackage[latin1]{inputenc}
\usepackage[active]{srcltx}
\usepackage{algorithm}
\usepackage[noend]{algorithmic}
\usepackage{amsmath}
\usepackage{amssymb}
\usepackage{amsthm}
\usepackage{bbm}
\usepackage{enumerate}
\usepackage{graphicx}
\usepackage{ifthen}
\usepackage{listings}
\usepackage{struktex}
\usepackage{hyperref}
\usepackage[breakable]{tcolorbox}
\usepackage[a4paper, left=2cm, right=2cm, top=2cm]{geometry}
\usepackage{mathtools}
\usepackage{tikz}
\usepackage{dsfont}
\usepackage{multicol}
\usepackage{pgfplots}
\usetikzlibrary{trees}
\pgfplotsset{compat=newest}
\usetikzlibrary{shapes.geometric}


\newcommand{\contradiction}{{\hbox{%
			\setbox0=\hbox{$\mkern-3mu\times\mkern-3mu$}%
			\setbox1=\hbox to0pt{\hss$\times$\hss}%
			\copy0\raisebox{0.5\wd0}{\copy1}\raisebox{-0.5\wd0}{\box1}\box0
}}}

%%%%%%%%%%%%%%%%%%%%%%%%%%%%%%%%%%%%%%%%%%%%%%%%%%%%%%
\newcommand{\Fach}{Analytische Geometrie}
\newcommand{\Semester}{SoSe 21}
\newcommand{\Uebungsblatt}{Abi 2012/2013 GK Nachtermin: Aufgabe B2} 
\newcommand{\nl}{\\[0,20cm]}
\newcommand{\lnl}{\\[0,30cm]}
\newcommand{\xlnl}{\\[0,75cm]}
%%%%%%%%%%%%%%%%%%%%%%%%%%%%%%%%%%%%%%%%%%%%%%%%%%%%%%


\setlength{\parindent}{0em}
\topmargin -2.0cm
\oddsidemargin 0cm
\evensidemargin 0cm
\setlength{\textheight}{9.6in}
\setlength{\textwidth}{6.9in}
\addtolength{\hoffset}{-22pt}


\newcommand{\limes}[2]{
	\lim\limits_{x\rightarrow #1}\quad  #2
}
\newcommand{\limesh}[1]{
	\lim\limits_{h\rightarrow 0}\quad  #1
}
\newcommand{\limesr}[2]{
	\lim\limits_{\underset{x > #1}{x\rightarrow #1}}\quad  #2
}
\newcommand{\limesl}[2]{
	\lim\limits_{\underset{x < #1}{x\rightarrow #1}}\quad  #2
}
\newcommand{\Aufgabe}[2]{
	{
		\vspace*{0.3cm}
		\begin{tcolorbox}[breakable,colback=yellow!0,colframe=black!65!black,title=\textbf{Aufgabe #1:},width=\linewidth ]
			{#2}
		\end{tcolorbox}
		
		
	}
}
\newcommand{\Hinweis}[1]{
	\vspace*{0.3cm}
	\begin{tcolorbox}[breakable,colback=yellow!10,colframe=yellow!65!black,title=\textbf{Hinweis:},width=\linewidth ]
		{#1}
	\end{tcolorbox}
}
\newcommand{\SHA}[1]{
	\vspace*{0.1cm}
	\begin{tcolorbox}[breakable,colback=blue!5,colframe=blue!65!black,title=\textbf{Richtiger SHA256 Hash:},width=\linewidth ]
		{\texttt{{#1}}}
	\end{tcolorbox}
}
\newcommand{\Beispiel}[1]{
	\vspace*{0.2cm}
	\begin{tcolorbox}[breakable,colback=yellow!0,colframe=green!65!black,title=\textbf{Beispiel:},width=\linewidth ]
		{#1}
	\end{tcolorbox}
}
\newcommand{\p}[2]{\pi_{#2}^{(#1)}}
\newcommand{\eing}[1]{\begin{enumerate}[\quad]
		\item #1
\end{enumerate}}

\newcommand{\abc}[1]{
	\begin{enumerate}[(a)]
		#1
	\end{enumerate}
}
\newcommand{\vectr}[3]{\begin{pmatrix}#1\\#2\\#3
\end{pmatrix}}
\newcommand{\integral}[4]{\int\limits_{#1}^{#2} {#3} {\quad d #4}}
\newcommand{\summe}[3]{\sum\limits_{#1}^{#2} #3}
\begin{document}
	\thispagestyle{fancy}
	\pagestyle{fancy}
	\begin{center}
		\LARGE \sf \textbf{ \Uebungsblatt{}}
	\end{center}
	\vspace*{0.1cm}
	\tableofcontents
	\newpage
	\thispagestyle{fancy}
	\setcounter{section}{1}
	\section{Aufgabe B2}
	\subsection{Längen der Spannseile von Punkt B}
		\begin{align*}
			\vec{BD}&= D - B = \vectr{-40}{-80}{65}\\
			\vec{BE}&= D - E = \vectr{-40}{-80}{116}\\\\
			\implies \left|\vec{BD}\right| &= \sqrt{ (-40)^2 + (-80)^2 + (65)^2}=110.57\\
			\implies \left|\vec{BD}\right| &= \sqrt{ (-40)^2 + (-80)^2 + (116)^2}=146.48
		\end{align*}
	\newpage
	\subsection{A und C bestimmen}
		\subsubsection{C bestimmen}
		\[H: x+20z=0,\qquad g: \vec{x}=\vec{E}+ k \cdot \vec{v}=\vectr{0}{0}{114} + k\cdot \vectr{5}{-10}{14}\]
		Schnittpunkt $g$ mit $H$ ausrechnen: $g$ in $H$ einsetzten:
		\begin{align*}
			x+20z&=0\\
			5k + 20(114 + k\cdot 14) &= 0&&\\
			5k + 2280 + 280k &= 0&&\\
			285k + 2280&=0&& \vert \,-2280\\
			285k &=- 2280 &&\vert \,:285\\
			k&=-8 
		\end{align*} 
		$k$ einsetzten und schnittpunkt ausrechnen:
		\begin{align*}
			\vec{OC}&=\vectr{0}{0}{114} -8\cdot \vectr{5}{-10}{14}=\vectr{0}{0}{114} + \vectr{-40}{80}{-112}\\\\
			&=\vectr{-40}{80}{2}\\
			\implies C&=(-40|\,80|\,2)
		\end{align*}
		\subsubsection{A bestimmen}
		\[A=(0,y,0),\qquad \left|\vec{EA}\right| = 145,\qquad y\in \mathds{R}^-\]
		Gleichung:
		\begin{align*}
			\left|\vec{EA}\right| &= 145&&\\
			\sqrt{(0)^2+(y)^2+(-114)^2}&= 145 &&\vert\,\,  ^2 \\
			(y)^2+(-114)^2 &= 145^2&&\vert\, - (-114)^2 \\
			y^2 &= 145^2- (-114)^2 = 8029\\
			\implies y& = -\sqrt{8029}=-89.6\\\\
			\implies A=(0|\,-89.6|\,0)
		\end{align*}
	\newpage
	\subsection{Landwirtschaftliche Nutzfläche}
	\subsubsection{$\Delta OBD$ kein rechtwinkliges Dreieck}
		\[\vec{OB}= \vectr{40}{80}{-2},\quad \vec{OD}= \vectr{0}{0}{63},\quad  \vec{BD}= \vectr{-40}{-80}{65} \]
		
		Bestimmung des Winkels (über Skalarprodukt):
		\begin{align*}
			\vec{OB} \circ \vec{OD} &= 40\cdot 0 + 80 \cdot 0 - 2 \cdot 63\\
									&=- 126\not=0\\
									\implies \text{Kein rechter Winkel bei } &O\\\\
		\vec{OB} \circ \vec{BD} &= 40\cdot (-40) + 80 \cdot (-80) - 2 \cdot 65\\
		&=-8130\not=0\\
		\implies \text{Kein rechter Winkel bei } &B\\\\
		\vec{OD} \circ \vec{BD} &= 0\cdot (-40) + 0 \cdot (-80) + 63 \cdot 65\\
		&=4095\not=0\\
		\implies \text{Kein rechter Winkel bei } &D
	\end{align*}
	$\implies$ kein rechtwinkliges Dreieck $\Delta OBD$
	\subsubsection{Winkel $\beta$ zwischen $\vec{BO}$ und $\vec{BD}$}
		\[\vec{BO}= \vectr{-40}{-80}{2},\quad \vec{BD}= \vectr{-40}{-80}{65}\]
		Bestimmung des Winkels (über Skalarprodukt):
		\begin{align*}
			\frac{\vec{BO} \circ \vec{BD} }{|\vec{BO}| \cdot |\vec{BD}|}&= \frac{40^2 + 80^2 + 2\cdot 65}{\sqrt{40^2+80^2+2^2} \cdot \sqrt{40^2+80^2+65^2}}\\
			&= \frac{8130}{9891.86} = 0.8219\\\\
			\beta &= acos(0.8219)= cos^{-1}(0.8219) = 34.72^\circ
		\end{align*}
	\Hinweis{Dein Fehler war die beiden Vektoren in unterschiedliche Richtungen laufen zu lassen. Es wird immer der Winkel berechnet welcher in Richtung beider Vektoren anliegt. Wenn du von $180^\circ$ deinen Winkel abziehst wirst du auf das gleiche $\beta$ kommen wie ich. }
	\subsubsection{Abstand von $B$ und $F$}
		Pythagoras:
		\begin{align*}
			tan(\beta) &= \frac{4}{|\vec{BF}|}\\
			\Leftrightarrow |\vec{BF}|&=\frac{4}{tan(\beta)}\\
			&=5.77 m
		\end{align*}
	\subsubsection{Koordinaten Punkt F}
	\[\vec{BO}= \vectr{-40}{-80}{2},\quad |\vec{BF}|=5.77 m \]
	Über Gerade lösen:
	\begin{align*}
		\vec{OF} &= \vec{OB} + \frac{|\vec{BF}|}{|\vec{BO}|}\cdot \vec{BO}\\
			&=  \vectr{40}{80}{-2} + \frac{5.77}{\sqrt{40^2+80^2+2^2}}\cdot \vectr{-40}{-80}{2}\\
			&= \vectr{40}{80}{-2} + 0.06\cdot \vectr{-40}{-80}{2}\\
			&=\vectr{40}{80}{-2}+\vectr{-2.4}{-4.8}{0.12}\\
			&=\vectr{37.6}{77.6}{-1.88}\\\\
			\implies F&=(37.6|\,77.6|\,-1.88)
	\end{align*}
\newpage
\subsection{Sturm und Sendemast}
	\[g_1:\vec{x}=\vectr{100}{10}{-4}+ k \cdot  \vectr{-100.10}{- 10.20}{154.00} ,\qquad g_2:\vec{x}=\vectr{10}{100}{0.75} + t \cdot \vectr{-10.10}{-100.20}{149.25}\]
	Schnittpunkt $g_1$ und $g_2$:
	\begin{align*}
		\vectr{100}{10}{-4}+ k \cdot  \vectr{-100.10}{- 10.20}{154.00} &= \vectr{10}{100}{0.75} + t \cdot \vectr{-10.10}{-100.20}{149.25} && \vert \, -\vectr{100}{10}{-4}\\
	 k \cdot  \vectr{-100.10}{- 10.20}{154.00} &= \vectr{-90}{90}{4.75} + t \cdot \vectr{-10.10}{-100.20}{149.25} && \vert \, - t \cdot \vectr{-10.10}{-100.20}{149.25}\\
	 k \cdot  \vectr{-100.10}{- 10.20}{154.00}- t \cdot \vectr{-10.10}{-100.20}{149.25} &= \vectr{-90}{90}{4.75} && \\
	 \implies k=1, \quad & t=1&&\\\\
	 \text{Einsetzten ergibt:}&&&&\\
	 E' &= ( -0,1 |\, -0,2 |\, 150 )
	\end{align*}
	Winkel zwischen Sendemast vor und Nach dem Sturm ($\varphi$):
	\[\vec{OE'}=\vectr{-0.1}{-0.2}{150} ,\qquad \vec{OE}=\vectr{0}{0}{114} \] 
	\begin{align*}
		\frac{\vec{OE'} \circ \vec{OE} }{|\vec{OE'}| \cdot |\vec{OE}|}&= \frac{-0.1 \cdot 0 + -0.2 \cdot 0 + 150 \cdot 114}{\sqrt{(-0.1)^2+(-0.2)^2+(150)^2}\cdot \sqrt{(0)^2+(0)^2+(114)^2}}\\
		&=\frac{17100}{17100.02}=0.99999883\\\\
		 &\color{gray}\text{ Runden ist hier nicht sinnvoll, da } cos(1)=0^\circ \text{ ist.}\color{black} \\
		\varphi = acos(0.99999883) &= cos^{-1}(0.99999883) \approx 0.09^\circ 
	\end{align*}
	Damit ist der Mast auch nichtmehr senkrecht.
	\Hinweis{Theoretisch könnte man auch zeigen das der Mast nicht mehr senkrecht ist, indem du das Skalarprodukt mit den Einheitsvektoren $e_1=\vectr{1}{0}{0},\, e_2=\vectr{0}{1}{0}$ bildest. (Theoretisch müsste das aber auch über Winkel ($\varphi$) argumentierbar sein, denn $E$ war ja senkrecht, also ist $E'$ nicht mehr senkrecht wenn der Winkel größer als $0$ ist)}
\end{document} 
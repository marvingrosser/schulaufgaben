\documentclass[a4paper,12pt]{article}
\usepackage{fancyhdr}
\usepackage{fancyheadings}
\usepackage[ngerman,german]{babel}
\usepackage{german}
\usepackage[utf8]{inputenc}
%\usepackage[latin1]{inputenc}
\usepackage[active]{srcltx}
\usepackage{algorithm}
\usepackage[noend]{algorithmic}
\usepackage{amsmath}
\usepackage{amssymb}
\usepackage{amsthm}
\usepackage{bbm}
\usepackage{enumerate}
\usepackage{graphicx}
\usepackage{ifthen}
\usepackage{listings}
\usepackage{struktex}
\usepackage{hyperref}
\usepackage[breakable]{tcolorbox}
\usepackage[a4paper, left=2cm, right=2cm, top=2cm]{geometry}
\usepackage{mathtools}
\usepackage{tikz}
\usepackage{dsfont}
\usepackage{multicol}
\usepackage{pgfplots}
\usetikzlibrary{trees}
\pgfplotsset{compat=newest}

\tikzstyle{level 1}=[level distance=3.5cm, sibling distance=4.5cm]
\tikzstyle{level 2}=[level distance=3.5cm, sibling distance=2cm]



% Define styles for bags and leafs
\tikzstyle{bag} = [text width=2em, text centered]
\tikzstyle{end} = [circle, minimum width=3pt,fill, inner sep=1pt]

\newcommand{\contradiction}{{\hbox{%
			\setbox0=\hbox{$\mkern-3mu\times\mkern-3mu$}%
			\setbox1=\hbox to0pt{\hss$\times$\hss}%
			\copy0\raisebox{0.5\wd0}{\copy1}\raisebox{-0.5\wd0}{\box1}\box0
}}}

%%%%%%%%%%%%%%%%%%%%%%%%%%%%%%%%%%%%%%%%%%%%%%%%%%%%%%
\newcommand{\Fach}{Grenzwerte}
\newcommand{\Semester}{SoSe 21}
\newcommand{\Uebungsblatt}{Analysis I} 
\newcommand{\nl}{\\[0,20cm]}
\newcommand{\lnl}{\\[0,30cm]}
\newcommand{\xlnl}{\\[0,75cm]}
%%%%%%%%%%%%%%%%%%%%%%%%%%%%%%%%%%%%%%%%%%%%%%%%%%%%%%


\setlength{\parindent}{0em}
\topmargin -2.0cm
\oddsidemargin 0cm
\evensidemargin 0cm
\setlength{\textheight}{9.6in}
\setlength{\textwidth}{6.9in}
\addtolength{\hoffset}{-22pt}


\newcommand{\limes}[2]{
	\lim\limits_{x\rightarrow #1}\quad  #2
}
\newcommand{\limesh}[1]{
	\lim\limits_{h\rightarrow 0}\quad  #1
}
\newcommand{\limesr}[2]{
	\lim\limits_{\underset{x > #1}{x\rightarrow #1}}\quad  #2
}
\newcommand{\limesl}[2]{
	\lim\limits_{\underset{x < #1}{x\rightarrow #1}}\quad  #2
}
\newcommand{\Aufgabe}[2]{
	{
		\vspace*{0.3cm}
		\begin{tcolorbox}[breakable,colback=yellow!0,colframe=black!65!black,title=\textbf{Aufgabe #1:},width=\linewidth ]
			{#2}
		\end{tcolorbox}
		
		
	}
}
\newcommand{\Hinweis}[1]{
	\vspace*{0.3cm}
	\begin{tcolorbox}[breakable,colback=yellow!10,colframe=yellow!65!black,title=\textbf{Hinweis:},width=\linewidth ]
		{#1}
	\end{tcolorbox}
}
\newcommand{\SHA}[1]{
	\vspace*{0.1cm}
	\begin{tcolorbox}[breakable,colback=blue!5,colframe=blue!65!black,title=\textbf{Richtiger SHA256 Hash:},width=\linewidth ]
		{\texttt{{#1}}}
	\end{tcolorbox}
}
\newcommand{\Beispiel}[1]{
	\vspace*{0.2cm}
	\begin{tcolorbox}[breakable,colback=yellow!0,colframe=green!65!black,title=\textbf{Beispiel:},width=\linewidth ]
		{#1}
	\end{tcolorbox}
}
\newcommand{\p}[2]{\pi_{#2}^{(#1)}}
\newcommand{\eing}[1]{\begin{enumerate}[\quad]
		\item #1
\end{enumerate}}

\newcommand{\abc}[1]{
	\begin{enumerate}[(a)]
		#1
	\end{enumerate}
}

\newcommand{\integral}[4]{\int\limits_{#1}^{#2} {#3} {\quad d #4}}
\newcommand{\summe}[3]{\sum\limits_{#1}^{#2} #3}
\begin{document}
	\thispagestyle{fancy}
	\begin{center}
		\LARGE \sf \textbf{ \Uebungsblatt{}}
	\end{center}
	\vspace*{0.1cm}
	\Hinweis{
		Ankreuzaufgaben wie Aufgabe 1, Aufgabe 5 und ... kannst du selbst kontrollieren.
		Schreibe dafür die angekreuzten Buchstaben (in alphabetischer Reihenfolge) hintereinander und wandle sie in einen SHA256 - Hash um.\\ Du kannst dafür diese Seite benutzen:
		\href{https://xorbin.com/tools/sha256-hash-calculator}{https://xorbin.com/tools/sha256-hash-calculator}\\
		Danach vergleichst du die Antwort einfach mit dem gegebenem Hash-Wert. Wenn sie übereinstimmen waren deine Antworten richtig.
		\Beispiel{
			\begin{multicols}{2}
				\begin{enumerate}[(a)]
					\item Richtig
					\item Falsch
					\item Falsch
					\item Richtig
				\end{enumerate}
			\end{multicols}
		\SHA{70ba33708cbfb103f1a8e34afef333ba7dc021022b2d9aaa583aabb8058d8d67}
			Nun schreibe ich ''ad'' in das Eingabefeld der Webseite.
			Der zurückgegebene Hash müsste mit dem von oben übereinstimmen.
		}
		Achtung, allein eine falsche Antwort (also ein anderer, fehlender oder Buchstabe zu viel) führt zu einem falschen Ergebnis. Also schaue am besten bei welcher Aufgabe du dir am unsichersten bist.\\
		Außerdem rate ich davon ab alle Möglichkeiten blind durchzuprobieren. Die Anzahl der Möglichkeiten die du eingeben müsstest beträgt $\quad 2^{Aufgabenzahl} - 1$.\\
		Bei 8 Aufgaben sind das 255 Möglichkeiten.
	}\newpage
	%<<<
	\Aufgabe{1\quad Kreuze die richtigen Antworten an}{
		\begin{center}
			\textbf{Gegeben:} \qquad $f(x)=\frac{1}{x^2} + 1,\qquad g(x)=e^x, \qquad h(x)=3x^2 + 4$\\
		\end{center}
		\begin{multicols}{2}
			\begin{enumerate}[(a){\qed}]
				\item Der Wert der Ableitung einer Funktion zu $x$ ist ihr Differentenquotient an der Stelle $x$.
				\item $g(x)=g^{\prime}(x)=g^{\prime\prime}(x)=g^{\prime\prime\prime}(x)=...$
				\item $f(-2)=\frac{3}{4}$
				\item $f'(x)=\frac{1}{2x}$
				\item Wertebereich von h:\\  $W_{h}=\{x | x \in \mathds{R}, x\geq 4\}$
				\item $g(0)=1$
				\item $h$ hat 2 Nullstellen.
				\item $h(f(x))= \frac{1}{(3x^2+4)^2} + 1$
				\item Der Wert, der Ableitung einer Funktion, von $x$ ist ihr Differentialquotient an der Stelle $x$.
				\item Der Anstieg der Normale ($m_N$) zu einer gegebenen Tangente mit Anstieg ($m_T$) berechnet man mit $m_N= \frac{1}{m_T}$
				\item Definitionsbereich von f:\\ $D_{f}=\{x| x\in \mathds{R}, x>0\}$
				\item $g$ hat eine Waagerechte Asymptote\\
			\end{enumerate}
		\end{multicols}
	\SHA{8a77f5d07ec990de1802df20cca2e9f07bdb8267b19a175b536737a62675d982}
	}
	\Aufgabe{2\quad Gib die Ableitungen der gegebenen Funktionen an}{
		\begin{multicols}{2}
			\begin{enumerate}[(a)]
				\item $\quad f(x)=4x^2 + 3$
				\item $\quad f(x)=sin(x)$
				\item $\quad f(x)=3e^x$
				\item $\quad f(x)=2x \cdot cos(x)$
				\item $\quad f(x)=e^{4x^2}$
				\item $\quad f(x)=3x^2 \cdot (x^2+3)^2$
				\item $\quad f(x)=x^2 - 6x$
				\item $\quad f(x)=ln(x)$
				\item $\quad f(x)=2x^4 + sin(x)$
				\item $\quad f(x)=\frac{1}{\sqrt{x}}$
				\item $\quad f(x)=6$
				\item $\quad f(x)= sin(2x)\cdot x^3$
			\end{enumerate}
		\end{multicols}
	}
	\Aufgabe{3\quad Zeichne die Ableitungsfunktion ein}{
			\begin{center}
				\begin{tabular}{c c}
					\begin{tikzpicture}
						\begin{axis}[
							xmin=-10,xmax=10,
							ymin=-6,ymax=8,
							grid=both,
							grid style={line width=.1pt, draw=gray!10},
							major grid style={line width=.2pt,draw=gray!50},
							axis lines=middle,
							minor tick num=4,
							enlargelimits={abs=0.5},
							axis line style={latex-latex},
							ticklabel style={font=\tiny,fill=white},
							xlabel style={at={(ticklabel* cs:1)},anchor=north west},
							ylabel style={at={(ticklabel* cs:1)},anchor=south west},
							samples=50
							]
							\addplot[blue, thick,domain=-10:10,](x,0.025*x*x*x - 2*x + 1);
						\end{axis}
					\end{tikzpicture}&
					\begin{tikzpicture}
						\begin{axis}[
							xmin=-10,xmax=10,
							ymin=-1,ymax=1,
							grid=both,
							grid style={line width=.1pt, draw=gray!10},
							major grid style={line width=.2pt,draw=gray!50},
							axis lines=middle,
							minor tick num=4,
							enlargelimits={abs=0.5},
							axis line style={latex-latex},
							ticklabel style={font=\tiny,fill=white},
							xlabel style={at={(ticklabel* cs:1)},anchor=north west},
							ylabel style={at={(ticklabel* cs:1)},anchor=south west},
							samples=100
							]
							\addplot[blue, thick,domain=-10:10,]{sin(deg(x))};
						\end{axis}
					\end{tikzpicture}	\\
					\begin{tikzpicture}
						\begin{axis}[
							xmin=-10,xmax=10,
							ymin=-10,ymax=10,
							grid=both,
							grid style={line width=.1pt, draw=gray!10},
							major grid style={line width=.2pt,draw=gray!50},
							axis lines=middle,
							minor tick num=4,
							enlargelimits={abs=0.5},
							axis line style={latex-latex},
							ticklabel style={font=\tiny,fill=white},
							xlabel style={at={(ticklabel* cs:1)},anchor=north west},
							ylabel style={at={(ticklabel* cs:1)},anchor=south west},
							samples=100
							]
							\addplot[blue, thick,domain=-10:10,](x,2*x-1);
						\end{axis}
					\end{tikzpicture}&	
					\begin{tikzpicture}
						\begin{axis}[
							xmin=-3,xmax=5,
							ymin=-10,ymax=10,
							grid=both,
							grid style={line width=.1pt, draw=gray!10},
							major grid style={line width=.2pt,draw=gray!50},
							axis lines=middle,
							minor tick num=4,
							enlargelimits={abs=0.5},
							axis line style={latex-latex},
							ticklabel style={font=\tiny,fill=white},
							xlabel style={at={(ticklabel* cs:1)},anchor=north west},
							ylabel style={at={(ticklabel* cs:1)},anchor=south west},
							samples=130
							]
							\addplot[blue, thick,domain=-10:10,](x,1/9 * x^4 -4/9*x^3 - 4/9*x^2 +16/9*x);
						\end{axis}
					\end{tikzpicture}	
				\end{tabular}
			\end{center}
	}
	\Aufgabe{4\quad Welche der Funktionen haben eine Asymptote $\mathbf{y=2}$\quad }{
		\begin{multicols}{4}
			\begin{enumerate}[(a)]
				\item $f(x)=\frac{x^3 + 2x +1}{3x^2 + 3x^4 + 1}$
				\item $f(x)=2+x^{-2}$
				\item $f(x)= \frac{2x^2+1}{1x-1}$
				\item $f(x)=\frac{2x^2 + 10x^3 }{-3x +5x^3}$
			\end{enumerate}
		\end{multicols}
	\SHA{5e657ff6158d3e2a6d23e2a523917a2305acee9423365e268695c4b7b8919f4c}
	}
	\Aufgabe{5\quad Finde die Tangente/Normale an der Stelle $\mathbf{x_0}$}{
	\begin{multicols}{3}
		\begin{enumerate}[(a)]
			\item Tangente:\\$\quad x_0=2,\\f(x)=x^2 - 4x +4$
			\item Normale:\\$\quad x_0=2,\\f(x)=x^2$
			\item Normale:\\$\quad x_0=1,\\f(x)= x^3 - x^2$
			\item Tangente:\\$\quad x_0=0,\\f(x)= 3x+1 $
			\item Tangente:\\$\quad x_0=-2,\\f(x)= (x^2+1)^2 $
			\item Tangente:\\$\quad x_0=-1,\\f(x)=  5x^2$
		\end{enumerate}
\end{multicols}
}
	\Aufgabe{6\quad Finde die Tangente/Normale mit dem gegebenem Anstieg}{
		\begin{multicols}{3}
			\begin{enumerate}[(a)]
				\item Tangente:\\$\quad m=2,\\f(x)=x^2$
				\item Tangente:\\$\quad m=0,\\f(x)= (x+1)^2 $
				\item Normale:\\$\quad m=2,\\f(x)=\frac{1}{4}x^2$
				\item Normale:\\$\quad m=\frac{1}{4},\\f(x)= x^3 +2x^2+ 1$
				\item Tangente:\\$\quad m=-7,\\f(x)= 2x^2 + x  $
				\item Tangente:\\$\quad m=0.25,\\f(x)= \frac{1}{4}x^4+\frac{1}{8}x^2$
			\end{enumerate}
		\end{multicols}
	}
	\Aufgabe{7\quad Finde die Tangenten die durch den Punkt Y gehen}{
		\begin{multicols}{2}
			\begin{enumerate}[(a)]
				\item $\quad Y=(0;4,25),\\f(x)=-x^2 + 1$
				\item $\quad Y=(-1;-\frac{32}{3}),\\f(x)=\frac{1}{3}x^3 - 2 x^2$
			\end{enumerate}
		\end{multicols}
	}

	\Aufgabe{8\quad Finde die Extrempunkte der gegebenen Funktion}{
		\begin{multicols}{2}
			\begin{enumerate}[(a)]
				\item $\quad f(x)=-x^2 + 6x - 6$
				\item $\quad f(x)=\frac{1}{3}x^3 - 9 x + 3$
				\item $\quad f(x)=x^3 + 2x^2 + x$
				\item $\quad f(x)=cos(x)$\\
					($x$ in Grad)
			\end{enumerate}
		\end{multicols}
	}
	\Aufgabe{9\quad Gib das Monotonieverhalten der Funktionen an}{
		\begin{multicols}{2}
			\begin{enumerate}[(a)]
				\item $f(x)= x^3 - 27x + 2$
				\item $f(x)=x^2 - 4x$
				\item $f(x)=\frac{1}{9}  x^4 -\frac{4}{9}x^3 - \frac{4}{9}x^2 +\frac{16}{9}x$
				\item $f(x)=sin(x)$
			\end{enumerate}
		\end{multicols}
	}
	\Aufgabe{10\quad Anwendungsaufgaben}{
		\begin{enumerate}[(I)]
			\item Wegen der prekären COVID-19 Situation soll in Wuhan ein neues Krankenhaus gebaut werden. Pro Raum der Intensivstation stehen nur begrenzte Materialien zur Verfügung. Desshalb steht pro Raum nur 20 m Wand zur Verfügung.
			Welche Seitenlängen müssen die Räume haben, damit pro Raum möglichst viele Intensivbetten Platz haben, also die Raumfläche maximiert wird?\\\\
			Flächeninhalt Rechteck:
			\[A_R= a\cdot b\]
			Umfang Rechteck:
			\[U_R = 2a + 2b\]
			Geben sie Haupt und Nebenbedingung an.
			\item Im Sportunterricht steht nächste Woche eine Kugelstoßen-LK an. Da Sie clever sind wissen sie, dass die Wurfweite ($s_w$) auch vom Abwurfwinkel ($\alpha$) abhängt:
			\[s_w= \frac{v_0^2 \cdot sin(2\alpha)}{9,81 \frac{m}{s^2}}\]
			Naturlich hängt die Weite auch von der Abwurfgeschwindigkeit $v_0$ ab.\\
			Sie haben in der Pause ein wenig experimentiert und herausgefunden, dass die Abwurfgeschwindigkeit so mit dem Winkel zusammenhängt:
				
			\[sin(\alpha) = \frac{10}{v_0 - 3} - 1\]
			In welchem Winkel werden Sie stoßen müssen, damit sie die bestmögliche Weite erreichen?\\
			Geben Sie Hauptbedingung und Nebenbedingung an.
			
		\end{enumerate}
	}

\end{document} 


\documentclass[a4paper,12pt]{article}
\usepackage{fancyhdr}
\usepackage{fancyheadings}
\usepackage[ngerman,german]{babel}
\usepackage{german}
\usepackage[utf8]{inputenc}
%\usepackage[latin1]{inputenc}
\usepackage[active]{srcltx}
\usepackage{algorithm}
\usepackage[noend]{algorithmic}
\usepackage{amsmath}
\usepackage{amssymb}
\usepackage{amsthm}
\usepackage{bbm}
\usepackage{enumerate}
\usepackage{graphicx}
\usepackage{ifthen}
\usepackage{listings}
\usepackage{struktex}
\usepackage{hyperref}
\usepackage[breakable]{tcolorbox}
\usepackage[a4paper, left=2cm, right=2cm, top=2cm]{geometry}
\usepackage{mathtools}
\usepackage{tikz}
\usepackage{dsfont}
\usetikzlibrary{trees}


\tikzstyle{level 1}=[level distance=3.5cm, sibling distance=4.5cm]
\tikzstyle{level 2}=[level distance=3.5cm, sibling distance=2cm]



% Define styles for bags and leafs
\tikzstyle{bag} = [text width=2em, text centered]
\tikzstyle{end} = [circle, minimum width=3pt,fill, inner sep=1pt]

\newcommand{\contradiction}{{\hbox{%
			\setbox0=\hbox{$\mkern-3mu\times\mkern-3mu$}%
			\setbox1=\hbox to0pt{\hss$\times$\hss}%
			\copy0\raisebox{0.5\wd0}{\copy1}\raisebox{-0.5\wd0}{\box1}\box0
}}}

%%%%%%%%%%%%%%%%%%%%%%%%%%%%%%%%%%%%%%%%%%%%%%%%%%%%%%
\newcommand{\Fach}{Grenzwerte}
\newcommand{\Semester}{SoSe 21}
\newcommand{\Uebungsblatt}{ Grenzwerte} 
\newcommand{\nl}{\\[0,20cm]}
\newcommand{\lnl}{\\[0,30cm]}
\newcommand{\xlnl}{\\[0,75cm]}
%%%%%%%%%%%%%%%%%%%%%%%%%%%%%%%%%%%%%%%%%%%%%%%%%%%%%%


\setlength{\parindent}{0em}
\topmargin -2.0cm
\oddsidemargin 0cm
\evensidemargin 0cm
\setlength{\textheight}{9.6in}
\setlength{\textwidth}{6.9in}
\addtolength{\hoffset}{-22pt}


\newcommand{\limes}[2]{
	\lim\limits_{x\rightarrow #1}\quad  #2
}
\newcommand{\limesh}[1]{
	\lim\limits_{h\rightarrow 0}\quad  #1
}
\newcommand{\limesr}[2]{
	\lim\limits_{\underset{x > #1}{x\rightarrow #1}}\quad  #2
}
\newcommand{\limesl}[2]{
	\lim\limits_{\underset{x < #1}{x\rightarrow #1}}\quad  #2
}
\newcommand{\Aufgabe}[2]{
	{
		\vspace*{0.3cm}
		\begin{tcolorbox}[breakable,colback=yellow!0,colframe=black!65!black,title=\textbf{Aufgabe #1:},width=\linewidth ]
			{#2}
		\end{tcolorbox}
		
		
	}
}
\newcommand{\Beispiel}[1]{
	\vspace*{0.3cm}
	\begin{tcolorbox}[breakable,colback=yellow!0,colframe=green!65!black,title=\textbf{Beispiel:},width=\linewidth ]
		{#1}
	\end{tcolorbox}
}
\newcommand{\p}[2]{\pi_{#2}^{(#1)}}
\newcommand{\eing}[1]{\begin{enumerate}[\quad]
		\item #1
\end{enumerate}}

\newcommand{\abc}[1]{
	\begin{enumerate}[(a)]
		#1
	\end{enumerate}
}

\newcommand{\integral}[4]{\int\limits_{#1}^{#2} {#3} {\quad d #4}}
\newcommand{\summe}[3]{\sum\limits_{#1}^{#2} #3}
\begin{document}
	\thispagestyle{fancy}
	\begin{center}
		\LARGE \sf \textbf{ \Uebungsblatt{}}
	\end{center}
	\vspace*{0.1cm}
	
	%<<<
	\Aufgabe{1 (Differentenquotient)}{
		\Beispiel{
			 \[f(x)=(x-1)^3 + 3 \qquad \text{ im Intervall }\quad [1;4]\]
			 
			 Anstieg: \begin{align*}
			 	m&= \frac{f(b)-f(a)}{b-a}=\frac{f(4)-f(1)}{4-1}\\
			 	&= \frac{((4-1)^3 + 3)-((1-1)^3 + 3)}{3}=\frac{((3)^3 + 3)-((0)^3 + 3)}{3}\\
			 	&=\frac{(27 + 3)-(3)}{3}=\frac{27}{3}=9
			 \end{align*}
		 Antwort: Der Anstieg der Funktion $f$ im Intervall $[1;4]$ beträgt $9$.
		}
		\vspace{1cm}
		\abc{
			\item \[f(x)=x^2 - 3 \qquad \text{ im Intervall }\quad [0;3]\]
			\item \[f(x)=x^5-3x^3 + 2x^2 -x + 7,5  \qquad \text{ im Intervall }\quad [-1;1]\]
			\item \[f(x)=\sqrt{x} \qquad \text{ im Intervall }\quad [4;6,25]\]
			\item \[f(x)=\frac{x+3}{x-2} \qquad \text{ im Intervall }\quad [3;4]\]
		}
	}
\newpage
	\Aufgabe{2 (Differentialquotient)}{
		\Beispiel{
			\[f(x)=x^2 -2 \qquad \text{ An der Stelle: }\quad x_0=-2\]
			Anstieg:
			\begin{align*}
				m&=\limesh{\frac{f(x_0 + h)-f(x_0)}{h}}=\limesh{\frac{f(-2 + h)-f(-2)}{h}}\\\\
				&= \limesh{\frac{(-2+h)^2 -2 -((-2)^2 - 2)}{h}}\\\\ &\overset{2. Binom. Formel}{=} \limesh{\frac{(-2)^2 -4h + h^2 -2 -((-2)^2 - 2)}{h}}\\\\
				&= \limesh{\frac{4 -4h + h^2 -2 -(4 - 2)}{h}} = \limesh{\frac{4 -4h + h^2 -2 -2}{h}}\\\\
				&= \limesh{\frac{ -4h + h^2 }{h}} = \limesh{\frac{ h\cdot(-4 + h) }{h}}\\\\
				&= \limesh{\frac{ \not h \cdot(-4 + h) }{\not{h}}} = \limesh{-4 + h}= -4\\
			\end{align*}
		Antwort: Der Anstieg der Funktion $f$ an der Stelle $-2$ beträgt $-4$.
			}
		\abc{
			\item \[f(x)=x^2 + 3 \qquad \text{ An der Stelle: }\quad x_0=-4\]
			\item \[f(x)=2x^3 \qquad \text{ An der Stelle: }\quad x_0=2\]
			\item \[f(x)=6x+1 \qquad \text{ An der Stelle: }\quad x_0=x\]
			\item \[f(x)=-4x^4-1 \qquad \text{ An der Stelle: }\quad x_0=3\]
		}
	}
\newpage
	\Aufgabe{3 (Stetigkeit/Unstetigkeit)}{
		\Beispiel{
			\[f(x)=\frac{1}{x}\]
			Unstetigkeitsstelle: \quad $x_0 = 0$,\quad da \quad $\implies \frac{1}{0}= undefiniert$ \\
			Betrachtung der Unstetigkeitsstelle $x_0$:\\\\
			$\circ\quad $Linksseitig:
			\begin{align*}
				\limesl{0}{f(x)}&=\limesl{0}{\frac{1}{x}}\qquad\implies \text{Werte nah an 0 aber kleiner als 0 einsetzen} \\\\
				f(-0.01)&=\frac{1}{-0.01}=-\frac{100}{1}= -100\qquad\qquad\quad\qquad\text{Nah an 0}\\
				f(-0.0001)&=\frac{1}{-0.0001}=-\frac{10000}{1}= -10000\qquad\qquad\text{Noch näher an 0}\\
				&\vdots\\\\
				\implies \limesl{0}{\frac{1}{x}} &= -\infty
			\end{align*}
			$\circ\quad $Rechtsseitig:
			\begin{align*}
				\limesr{0}{f(x)}&=\limesr{0}{\frac{1}{x}}\qquad\implies \text{Werte nah an 0 aber größer als 0 einsetzen} \\\\
				f(0.01)&=\frac{1}{0.01}=\frac{100}{1}= 100\qquad\qquad\quad\qquad\text{Nah an 0}\\
				f(0.0001)&=\frac{1}{0.0001}=\frac{10000}{1}= 10000\qquad\qquad\text{Noch näher an 0}\\
				&\vdots\\\\
				\implies \limesr{0}{\frac{1}{x}} &= +\infty\\
			\end{align*}
		Da beide Grenzwerte gegen $+$ bzw $- \infty$  streben und $f(x_0)=undef$, ist an der Stelle $x_0=0$ eine Polstelle.
		}\newpage
	Untersuche von den folgenden Funktionen die Definitionslücken (Stelle $x_0$, Art der
	Definitionslücke und ggf. Grenzwert). Alles ohne GTR/CAS
		\abc{
			\item \[f(x)=\frac{1}{1-x^2}\]
			\item \[f(x)=\frac{2}{2-x}\]
			\item \[f(x)=\frac{x^2-9}{3+x}\]
			\item \[f(x)=\frac{3x}{|x|}\]
			\item \[f(x)=\frac{x^2-6x+ 9
			}{x-3}\]
			\item \[f(x)=\frac{\frac{1}{3}x^2+ 2x + 3
			}{x+3}\]
			\item \[f(x)=\begin{cases}
				3x+6, & \text{für}\quad x>-1 \\
				x^2+2, & \text{für}\quad x<-1
			\end{cases}\]
			\item \[f(x)=\begin{cases}
				2x^2+2, & \text{für}\quad x<2 \\
				-x^2+4, & \text{für}\quad x>2
			\end{cases}\]

		}
	}

\end{document} 

